\documentclass[10pt,letterpaper,onecolumn,draftclsnofoot]{IEEEtran}
\usepackage[margin=0.75in]{geometry}
\usepackage{listings}
\usepackage{color}
\usepackage{longtable}
\usepackage{tabu}
\definecolor{dkgreen}{rgb}{0,0.6,0}
\definecolor{gray}{rgb}{0.5,0.5,0.5}
\definecolor{mauve}{rgb}{0.58,0,0.82}

\lstset{frame=tb,
  language=C,
  columns=flexible,
  numberstyle=\tiny\color{gray},
  keywordstyle=\color{blue},
  commentstyle=\color{dkgreen},
  stringstyle=\color{mauve},
  breaklines=true,
  breakatwhitespace=true,
  tabsize=4
}

\begin{document}
\begin{titlepage}
  \title{CS 461 - Fall 2016 - Problem Statement}
  \author{Jacob Broderick, Kristen Patterson, Brandon Chatham}
  \date{October 4, 2016}
  \maketitle
  \vspace{4cm}
  \begin{abstract}
  	\noindent 
  	  	\noindent 
  	Artificial Intelligence has solved many problems for games that involve taking turns, however there has not been much in the way of games played in real time. This project is to create an agent to play the game Starcraft Brood War, a real time strategy game created by Blizzard Entertainment. The end goal is to enter and compete in a competition using an agent created by Oregon State University students. After this project is complete, it could be expanded upon by future students in a club setting. This will be a template so that future students can strive to develop better solutions than those developed within this project.
  \end{abstract}
\end{titlepage}
\section{Introduction}
Gaming AI started as a way for game companies to give more life to the characters that appear in their game. These AI would interact with the user and would act dynamically based on the user's actions. Then researchers decided to try and fill the role of the user with an AI. This process was a lot more difficult as users had less constraints and more actions available compared to preprogrammed characters. Due to the difficulty of making an accurate AI, competitions emerged in order to provide researchers with a method to test and compare their findings against each other.

\section{Problem Definition}
Artificial intelligence has branched out into the gaming industry and many games are trying to be automated to play. With the success of AIs in games like chess and go, researchers have started to create AI for more difficult to automate games such as real time strategy games. However, due to the difficulty of making an accurate AI, it is extremely helpful to give a good starting example. Thus, we will make a sandbox designed for artificial intelligence for future clubs that may want to enter in a competition. The artificial intelligence will be focused around the game called Starcraft Brood War. The sandbox will essentially be some core mechanics and code taken from our attempt to create our own artificial intelligence. We will use this starting example to make an actual working artificial intelligence, but the important part that we must provide is a strong and easily reusable base for others to quickly pick up and understand. After completing the project, we will also be providing our results as a way for others to test and compare with their own AI to see if they are on the right track. The program should also be a good base for a future club to quickly and accurately familiarize themselves with the BWAPI. The BWAPI, also known as Brood War Application Programming Interface, is the provided interface for users to interact and import their artificial intelligence into the game. It also lets the coders access the information stored in Starcraft's code so they can use that information in their algorithms.

\section{Proposed Solution}
Our base objective is to create an artificial intelligence agent to perform a rudimentary script in a Starcraft tournament against other AI opponents. In order to do this, we will be using BWAPI to interact with our AI agent. In addition to the technical expectations, our team will be doing extensive documentation. Our code is expected to be exceedingly modular, allowing future users to take only pieces they may need to apply them to their projects with minimal adjustments necessary. Furthermore, our documentation should be thorough enough such that a developer would be able to understand how to use our code without any outside resources. This will fulfill the desire for an environment to support a future artificial intelligence club at OSU. At expo, we would like to present the replay of our competition and explain in more detail how exactly we have designed our agent. A stretch goal would be for us to design and implement an algorithm that can adjust to different scripted strategies based on the current state of the game. 

\section{Performance Metrics}
The project needs to perform the following metrics in order to be complete. An agent must be made that can follow a specific script and can be entered into a tournament for Starcraft Brood War. There must be multiple scripts that the agent can choose from. The agent must be able to decide if it needs to change script at all based on current or previous match data. The agent must not slow down the game by having 1 frame longer than 10 seconds, 10 frames longer than 1 second, or 320 frames longer than 55 milliseconds as per the tournament rules. The bot must be a runnable .exe or .dll file that can run on a Windows 7 32-bit machine. These requirements are from the BWAPI tournament that is held every spring.

\section{Conclusion}
Following our proposed solution, we will have completed the client's criteria and thus completed our problem. We will also have a replay from our competitions that we entered so we can display it for expo. We will know our project is successful by whether our client is pleased with the final product that we provide for him. Our client will then have his requested template for future clubs to use to help build their own AI.

\end{document}
