\documentclass[10pt,letterpaper,onecolumn,draftclsnofoot]{IEEEtran}
\usepackage[margin=0.75in]{geometry}
\usepackage{listings}
\usepackage{color}
\usepackage{longtable}
\usepackage{tabu}
\definecolor{dkgreen}{rgb}{0,0.6,0}
\definecolor{gray}{rgb}{0.5,0.5,0.5}
\definecolor{mauve}{rgb}{0.58,0,0.82}

\lstset{frame=tb,
  language=C,
  columns=flexible,
  numberstyle=\tiny\color{gray},
  keywordstyle=\color{blue},
  commentstyle=\color{dkgreen},
  stringstyle=\color{mauve},
  breaklines=true,
  breakatwhitespace=true,
  tabsize=4
}

\begin{document}
\begin{titlepage}
  \title{CS 461 - Fall 2016 - Problem Statement}
  \author{Jacob Broderick, Kristen Patterson, Brandon Chatham}
  \date{October 4, 2016}
  \maketitle
  \vspace{4cm}
  \begin{abstract}
  	\noindent 
  	Artificial Intelligence is a broad problem with many different applications. This project is to create an agent to play the game Starcraft Brood War. Starcraft Brood War is a real time strategy game created by Blizzard Entertainment. The end goal is to enter and compete in a competition using an agent created by Oregon State University students. After this project is complete, it could be expanded upon by future students in a club setting.The first step is implementing a scripted strategy. This is initially just building one offensive unit, or something that can be generalized. The next step is to have multiple scripts for the AI to choose from. The team would write a decision tree that takes in abstract information about the state of the game and decides if switching strategies would be more beneficial to the game.
  	
  	We are testing our progress in steps and checking with our client about the quality at each step. The final product is put through various situations and is compared to already existing artificial intelligence algorithms in a Starcraft AI competition. This fulfills the client's want for a general template for any future OSU clubs to use in order to create AI's for competitions or research. Our stretch goal is to adjust our scripted strategy based on live game data and previous match data if it is not a best-of-one series. 
  \end{abstract}
\end{titlepage}
\section{Introduction}
This document outlines the project Dr. Fern wants completed. Our client's field of study and research pertains to artificial intelligence and machine learning. The project that Dr. Fern wants completed is an artificial intelligence that plays Starcraft. This document is intended to clarify the problem to the development team as well as inform the client about how his problem will be solved.

\section{Problem Definition}
There is a constant drive to research and develop more artificial intelligence in order to further our knowledge. In order to expand the knowledge base, artifical intelligence is researched in a variety of fields from mathematics to video games. However, due to the difficulty of making an accurate AI, it is extremely helpful to give a good starting example. Thus, our client wants a template designed for artificial intelligence for future clubs that may want to enter in a competition. The artificial intelligence will be focused around the game called Starcraft Brood War. The template will essentially be some core mechanics and code taken from our attempt to create our own artificial intelligence. We will use this starting example to make an actual working artificial intelligence, but the important part that our client wants is a strong and easily reusable base for others to quickly pick up and understand. After completing the project, we will also be providing our results as a way for others to test and compare with their own AI to see if they are on the right track. The program should also be a good base for a future club to quickly and accurately familiarize themselves with the BWAPI. The BWAPI, also known as Brood War Application Programming Interface, is the provided interface for users to interact and import their artificial intelligence into the game. It also lets the coders access the information stored in Starcraft's code so they can use that information in their algorithms.

\section{Proposed Solution}
Our base objective is to create a sandbox for learning and experimentation focused on artificial intelligence with regards to Starcraft. The objective also includes the development of a rudimentary AI agent capable of competing in a Starcraft tournament against other AI opponents. In order to do this, we will be using our sandbox, essentially, building a basic product for the club our sandbox will be used for in the future. Furthermore, by the end of the project, the agent should be able to execute a scripted strategy. This will fulfill Dr. Fern's desire for an environment to support a future artificial intelligence club at OSU. At expo, we would like to present the replay of our competition and explain in more detail how exactly we have designed our agent and created our sandbox.
Creating a sandbox for the BWAPI is important because it allows us and future developers, full control over our own environment which our agent will run in. Essentially, we are developing both the agent, and the game environment the agent operates within. Using this, we can design and execute the algorithms necessary to making a dynamic AI. But more specifically to the competition, our agent will be optimized to the specifications of the competition standards that can be found in our performance metrics. 

\section{Performance Metrics}
The project needs to perform the following metrics in order to be complete. An agent must be made that can follow a specific script and can be entered into a tournament for Starcraft Brood War. There must be multiple scripts that the agent can choose from. The agent must be able to decide if it needs to change script at all based on current or previous match data. The agent must not slow down the game by having 1 frame longer than 10 seconds, 10 frames longer than 1 second, or 320 frames longer than 55 milliseconds as per the tournament rules. The bot must be a runnable .exe or .dll file that can run on a Windows 7 32-bit machine. These requirements are from the BWAPI tournament that is held every spring.

\section{Conclusion}
Following our proposed solution, we will have completed the client's criteria and thus completed our problem. We will also have a replay from our competitions that we entered so we can display it for expo. We will know our project is successful by whether our client is pleased with the final product that we provide for him. Our client will then have his requested template for future clubs to use to help build their own AI.

\end{document}
