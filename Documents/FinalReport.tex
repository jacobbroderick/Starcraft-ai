\documentclass[10pt,letterpaper,onecolumn,draftclsnofoot]{IEEEtran}
\usepackage[margin=0.75in]{geometry}
\usepackage{listings}
\usepackage{color}
\usepackage{longtable}
\usepackage{tabu}
\usepackage{pgfgantt}
\definecolor{dkgreen}{rgb}{0,0.6,0}
\definecolor{gray}{rgb}{0.5,0.5,0.5}
\definecolor{mauve}{rgb}{0.58,0,0.82}
\newcolumntype{P}{>{\centering\arraybackslash}m{6cm}}
\newcolumntype{L}{>{\centering\arraybackslash}m{3cm}}
\newcolumntype{C}{>{\centering\arraybackslash}m{3cm}}
\newcolumntype{B}{>{\centering\arraybackslash}m{1.5cm}}
\newcolumntype{M}{>{\centering\arraybackslash}m{2.5cm}}

\lstset{frame=tb,
	language=C,
	columns=flexible,
	numberstyle=\tiny\color{gray},
	keywordstyle=\color{blue},
	commentstyle=\color{dkgreen},
	stringstyle=\color{mauve},
	breaklines=true,
	breakatwhitespace=true,
	tabsize=4
}

\begin{document}
	
	\begin{titlepage}
		
		\title{CS 463 - Spring 2017 - Final Report \\ Project DevAI}
		\author{Jacob Broderick, Kristen Patterson, Brandon Chatham}
		\date{June 2, 2017}
		\maketitle
		\vspace{4cm}
		\begin{abstract}
			This document details the progress made by the Starcraft AI Project. It includes a brief recap of the project, where the project currently is, and what stretch goals remain unfinished. It also talks about any problems that have occurred. This document was made with a video that details similar information, but demonstrates what the working project looks like.
		\end{abstract}
	\end{titlepage}

\tableofcontents
\newpage
\section{Introduction}
	This project was requested by our client Dr. Fern. Dr. Fern's primary intent for this project was for us to reduce the learning curve of artificial intelligence and machine learning implementations for an AI club that will soon be created at OSU. Learning AI can be difficult but the basics can be picked up quickly. Some of the trouble spots for new developers who are not well-versed in the process of starteing projects might get hung up on configuring API's and setting up their development environment. Our project handles the configuration step for them. The environment is ready to use as soon as they open the Visual Studios project. Furthermore, if anything gets changed in a developer's settings, we have thorough documentation explaining how to fix it. Beyond this, we have included tools to help more skilled developers start data mining for machine learning algorithms with a python script we wrote and a replay analyzer we researched and chose from an open source developer. 
	There are three team members: Brandon Chatham, Kristen Patterson, and Jacob Broderick. Brandon primarily focused on code design, implementations, and overall direction of the project to best fit the desires of the client. Kristen focused on documentation. Nearly all of the documentation, even a small bit of the comments you will find in the code, are written by Kristen. Kristen worked closely with Brandon to keep up-to-date on what the newest implementations were in order to keep documentation relevant to the current state of the project. Jacob focused mostly on choosing our development tools in Visual Studios and BWAPI as well as researched other tools that may be valuable to us. BWAPI is the standard API used for Starcraft AI and was an easy choice. Determining what other tools were well-suited to the specifications of the project were slightly more difficult but fairly simple decisions. 
	Our client for the most part gave us autonomy on the project. While we did meet with him for required check-in's for the project, he allowed us to make our own decisions whether this was for better or for worse. We did consult him on a few occasions but usually made decisions primarily based on what the team felt was best based on what we knew Dr. Fern was most interested in.
\section{Original Requirements Document}

\section{Changes from Original Requirements Document}
\subsection{Final Gantt Chart}

\begin{figure}[h]
\begin{center}

\begin{ganttchart}[y unit title=0.4cm,
y unit chart=1.2cm,
vgrid,hgrid, 
title label anchor/.style={below=-1.6ex},
title left shift=.05,
title right shift=-.05,
title top shift = -.25,
title height=1,
bar/.style={fill=gray!50},
bar incomplete/.style={fill=red},
progress label text={},
bar height=0.7,
group right shift=0,
group top shift=.6,
group height=.3,
group peaks height=.2]
{1}{30}
%labels
\gantttitle{Terms}{30} \\
\gantttitle{Fall 2016}{10} 
\gantttitle{Winter 2017}{10} 
\gantttitle{Spring 2017}{10} 
%tasks
\ganttset{bar height=.4}
\ganttbar{Problem Statement}{3}{5} \\
\ganttbar{Requirements Document}{4}{6} \\
\ganttbar{Technical Document}{5}{8} \\
\ganttbar{Design Document}{5}{8} \\
\ganttbar{Learn BWAPI}{7}{10} \\
\ganttbar{Implementation}{11}{26} \\
\ganttbar{Stretch Goals}{23}{26} \\
\ganttbar{Prepare for Expo}{26}{27} \\
\ganttbar{Expo}{27}{28}
%relations
\ganttlink{elem0}{elem1}
\ganttlink{elem1}{elem2}
\ganttlink{elem1}{elem3}
\ganttlink{elem2}{elem5}
\ganttlink{elem3}{elem5}
\ganttlink{elem4}{elem5}
\ganttlink{elem5}{elem6}
\ganttlink{elem5}{elem7}
\ganttlink{elem6}{elem7}
\ganttlink{elem7}{elem8}

\end{ganttchart}
\end{center}
\caption{Final Gantt Chart}
\end{figure}
\section{Design Document}

\section{Tech Review}

\section{Weekly Blog Posts}
\subsection{Kristen Patterson}
Fall 3\\
Last week I had clarified the exact details of the project with my group mates and also contributed to the problem statement by revising the abstract, writing the problem definition, and writing both the introduction and some of the conclusion.\\ Planning with teammates about the requirements document and discussing problem statement with client.\\ Had some clarification issues with some details in the problem statement. Client was also on vacation so could not verify whether the document fit his criteria early enough.\\
\\
Fall 4\\
Last week our group had met with our client and had him look over our problem statement and provide feedback. I also corrected some grammatical and formatting errors with my section of the document. I also corrected the document based on the client's feedback.\\ Next week we need to get our client to sign our final draft of the problem statement. We also are going to have a rough draft of a requirements document by Wednesday to have ample time to set up a meeting to discuss with the client.\\ We had a problem when scheduling our meeting with the client as not all the group members were able to attend. I also had trouble figuring out why apostrophes would not show up properly on our tex document.\\
\\
Fall 5\\
Last week I had completed the introduction, formatting, gantt chart, and abstract of the requirements document.\\ Next week we will meet with our client and discuss the requirements document as well as revise any issues we have with our document.\\ We had trouble meeting with our client to get a signature as he was very busy. I also had some time management problems as it was a very busy week.\\
\\
Fall 6\\
Reviewed SRS document with client and made changes based on input.\\ Discuss with team about following documents and clarify materials needed with client.\\ Our team was fairly busy this week so there was a decent amount of time management problems.\\
\\
Fall 7\\
Got the final draft of the SRS document signed and submitted as well as clarified the technology review with our teachers and TA. Also broke our project down into 9 parts and split up the project.\\ Finish my portion of the tech review then make correction and meet with client to make sure it is acceptable.\\ I had time management problems as I had other assignments due. There was also some clarification issues on what our 9 parts of the tech review should be.\\
\\
Fall 8\\
Finished our Tech Review and submitted it. Discussed course of action for design document and progress reports.\\ Discuss format of final progress report and make any revisions to the tech review.\\ Mainly just time management problems again as this has been a busy term. Also need to clarify some questions for upcoming documents.\\
\\
Fall 9\\
Received revisions for Tech Review and discussed changes to them as well as assigned roles for the design document.\\ Complete my section of design document as well as my progress report and presentation.\\ No problems this week.\\
\\
Fall 10\\
Completed my section for the Design Document and Progress Report.\\ N/A\\ I have encountered some family issues that produced some problems in my schedule.\\
\\
Winter 1\\
Set up the environment to work on the project.\\ Start gathering modules for and practice using BWAPI.\\ Trouble getting a new schedule to meet with our team.\\
\\
Winter 2\\
Discussed timeline for project and started economy portion of code.\\ Continue working on economy portion.\\ Had a lot of other assignments that made it difficult to have more time for this project.\\
\\
Winter 3\\
Economy and Construction portion of code is finished.\\ Work on unit production and control.\\ More time issues and sickness as well.\\
\\
Winter 4\\
Started functions for training and selecting marines.\\ Start function for moving and attacking.\\ Just more time issues.\\
\\
Winter 5\\
Finished my portion of revisions and progress report and started code for alpha video.\\ Finish alpha video code and presentation.\\ I had 3 midterms so I had less time to focus on the project than I wanted.\\
\\
Winter 6\\
Finished all the revisions and progress report\\ Participate in elevator speech practice and start documentation of code\\ Time management\\
\\
Winter 7\\
Practiced elevator pitch for expo.\\ Start Documentation\\ Didn't get anything done as I was busy on other assignments.\\
\\
Spring 1\\
Wrote up documentation for scraper and scouting class.\\ Come up with visuals for the second draft of our poster.\\ Set up of new schedule with team.\\
\\
Spring 2\\
Discussed poster draft 2 with Kirsten for feedback and corrections and made revisions. Also found good visuals to use in poster.\\ Meet with group for group photo and meet and discuss poster with our client.\\ Was unsure of the best type of visuals to use for the poster.\\
\\
Spring 3\\
Completed poster draft 2 and discussed meeting time with client. Also completed flowchart for visual on poster.\\ Meet with client to discuss final poster and make corrections.\\ Had trouble trying to come up with visuals for the poster and the general structure but I ended up going to Kirsten and she helped.\\
\\
Spring 4\\
Discussed poster with client and made final corrections to code as well as discussed deliverables with our TA.\\ Make final corrections to the poster and submit it before the deadline.\\ We had some trouble trying to set up a time to meet with our client but because we had contacted him early we had plenty of leeway before the deadline.\\
\\
Spring 5\\
Completed Wired interview.\\ Prepare for Expo.\\ Had trouble trying to set up an appointment for the interview.\\
\\
Spring 6\\
Completed What we have so far portion of the Progress Report and the project details portion of the presentation.\\ Prepare for Expo.\\ Had trouble splitting up the work for the group.\\
\\
Spring 7\\
Completed attacking function in program and prepared video for expo as well as participated in Expo.\\ Class sweep up.\\ Tried to finish up everything quickly for expo.\\
\\
Final Post\\
If you were to redo the project from Fall term, what would you tell yourself?\\
I believe one of the biggest issues my team faced was that we got caught up in some of the more non-essential parts of the project and we didn't focus on completing the project before working on stretch goals and interesting mechanics.\\
\\
What's the biggest skill you've learned?\\
The biggest skill I have learned during this process is how to be able to talk about a project on multiple different levels of complexity.\\
\\
What skills do you see yourself using in the future?\\
The skills I have learned on how to explain my project and how to properly work in a team and manage my time appropriately.\\
\\
What did you like about the project, and what did you not?\\
I liked the concept and the coding of the project. It just felt like the schedule was catered more towards larger and more complex projects and I would have like to have more time focusing on development rather than extraneous planning.\\
\\
What did you learn from your teammates?\\
My teammates taught me a lot of technical skills that they were used to using and I believe I also helped them with it as well.\\
\\
If you were the client for this project, would you be satisfied with the work done?\\
I believe we provided a good starting point for students to use to supplement their research and learning.\\
\\
If your project were to be continued next year, what do you think needs to be working on?\\
The actual artificial intelligence portion of the code would be the focus as our project's main purpose was to supplement the development of that process.\\
\\
Speak a little about your expo experience.\\
I spoke with many individuals ranging from industry reps to people who did not really know what a video game was.\\
\subsection{Brandon Chatham}
Fall 3\\
Next week we plan on meeting with Dr. Fern to talk over our drafted problem statement. We did our best to write it this week with what we had gathered from our first meeting with him about what he imagines the project being. Hopefully, what we have is in-line with what he wants and he will sign-off on it.\\ This week we did some research on what is required for our competing in the Starcraft AI competition. We also started looking at some other agents created by other universities to see what other people have done. I familiarized myself with the BWAPI and did some research on what the benefits of our creating a sandbox for our project will be and why it was necessary. Sandbox development is a pretty basic and fundamental component to our project that I really had no familiarity with so I wanted to make sure I caught myself up to speed in that respect.\\ Our primary problem was Dr. Fern being out of the office this week, but that was not a big issue. Also, I still need to learn LaTeX which is not exactly a problem, but something I need to focus on. I had an issue trying to mess around with the BWAPI this week because in order to use it, I need to install a version of Stacraft Brood War which costs 15 dollars. I figured Dr. Fern might have a solution to that so I did not buy it, yet.\\
\\
Fall 4\\
This week we have been working on revising our problem statement and making sure it is up to par. Also, we met with all of our advising sources to sort out what those revisions might be. Tomorrow, they will be completed.\\ Next week we will start working on our next document. I need to start looking into the IEEE standard and familiarize myself with those standards.\\ We had to meet with Prof. Fern a bit later than we had hoped because he was out, but we should be up to speed.\\
\\
Fall 5\\
We worked on revising our problem statement. Additionally, Kristen made a gantt chart for us and we pulled together our requirements document rough draft.\\ Next week we will meet with our client to talk over what our requirements document should have.\\ I had trouble balancing my coursework with a large project in Operating Systems due and trying to get myself organized for the career fair. Also, we had some issues getting our document signed by our client but we sorted that out.\\
\\
Fall 6\\
We received our feedback on the problem statement and will hopefully talk over how we can improve it soon. Kristen worked more on refining the requirements document after getting feedback from Prof. Fern on what he felt needed to be changed.\\ Next week we will meet on Tuesday, begin working on other documents, and potentially fine-tune our problem statement.\\ I had some trouble getting adjusted to the formatting expectations of the IEEE document specifications and am still getting used to LaTeX.\\
\\
Fall 7\\
Started planning our work for the Tech Review document and talking about what technologies we will be talking about.\\ Writing our Tech Review for Monday and beginning to familiarize ourselves with the resources we will be working with for the project.\\ Not many problems other than two of our team members being sick but we have been able to keep up with the workload.\\
\\
Fall 8\\
We finished our work on the Tech Review.\\ Meet with group to start talking about our final project presentation and the design doc.\\ Trying to wrap my head around the IEEE format for the design doc and how I'll apply what I am focusing on for the project.\\
\\
Fall 9\\
Got feedback on our work for the Tech Review. We were not exactly thrilled with our overall grade but we plan to make some good revisions.\\ Meet with team to talk about our upcoming work for final presentation, design doc, and the tech review revisions.\\ Still trying to make sure I'm correctly understanding what is exactly expected for this next doc.\\
\\
Fall 10\\
Worked on and completed my design document portion.\\ Trying to find a way to work around Kristen's unfortunate family situation to get our final presentation done.\\
\\
Winter 1\\
I set up my machine with Starcraft Broodwar and BWAPI. I started fiddling around with some of the example modules and familiarized myself with how to start running the AI modules on our local machines.\\ Start writing code for the resource module which will allow us to start expanding into other modules once we have resources to spend.\\ Having some difficulties finding times for our group to meet and struggling to balance school with many job applications.\\
\\
Winter 2\\
Met with the team and started talking about our individual development schedules.\\ Writing code and pushing to our repository.\\ Same as last week\\
\\
Winter 3\\
Started getting development organized and got to the "starting line" for writing code and wrote code this week which we pushed.\\ Continue working on modules and utilizing them in our AI module. The resource module is nearly complete with basic functions to get the AI started. We will need to develop more sophisticated logic behind optimizing resource gathering in order to best suit our strategy and really, and economic desires. (The more resources the better in all scenarios as any improvements should improve every possible strategy.)\\ None\\
\\
Spring 1\\
Worked on the scraper implementation.\\ Talk with Kristen about the documentation for the scraper.\\ None.\\
\\
Spring 2\\
Worked on the scouting class a bit.\\ Work with the team on deciding our poster layout.\\ Finding time to meet with our client.\\
\\
Spring 3\\
Working on finalizing our project and making sure our expectations are met.\\ Meet with client and fix any minor bugs.\\ One persistent bug in our resource gathering class.\\
\\
Spring 4\\
Fixing the expansion bug.\\ Bring in a replay processor for our data mining.\\ None.\\
\\
Spring 5\\
Still need to fix the expansion bug.\\ Get our midterm report done and our pitch ready.\\ None.\\
\\
Spring 6\\
Tried to fix the expansion bug but were not able to.\\ Get our midterm report done and our pitch ready.\\ Couldn't fix the bug.\\
\\
Spring 7\\
If you were to redo the project from Fall term, what would you tell yourself?\\
To pay more attention to machine learning neural networks and how to fit that into the project.\\
\\
What's the biggest skill you've learned?\\
I have developed my skill to teach myself new technologies outside of the classroom setting which will be invaluable in the workplace setting.\\
\\
What skills do you see yourself using in the future?\\
My ability to convey my technical ideas and skills as well as my further-developed self learning skills.\\
\\
What did you like about the project, and what did you not?\\
I did not love how little guidance we received from our client but I did enjoy being able to learn through trial and error. Also finding ways to fix our mistakes once we realized we had made them.\\
\\
What did you learn from your teammates?\\
A lot about technical skills and how to best work together on such a large project.
\\
If you were the client for this project, would you be satisfied with the work done?\\
The package we developed is pretty decent and I think it could be a good starting point for new developers in the AI club he plans to start.\\
\\
If your project were to be continued next year, what do you think needs to be working on?\\
Definitely implementing neural networks that make our AI more dynamic. As is, we have created an interface to interact with it, but the neural network would take our project to the next level.\\
\\
Speak a little about your expo experience.\\
Expo was probably my least favorite part of the capstone project. While it was fun explaining the project to interested people, it was a bit of an elongated event in my opinion.
\section{Final Poster}

\newpage

\section{Project Documentation}
\subsection{Project Walkthrough}
The project is sectioned off into many classes for a library described below as well as an AI module. This AI module is the main location where the interaction with the game occurs. It has a startup phase where it converts the play map into data for the AI to use. After the initial startup phase the AI starts a nested for loop where it goes through each individual unit and gives it command. The worker is in charge of gathering materials, constructing buildings, and scouting. The Command Center is in charge of training workers. The barracks are in charge of training military. The marines are in charge of moving to a choke point and attacking the enemy. It does this loop each frame until a victor is decided. A victor is decided when on of the players buildings and enemies are all destroyed.
\subsection{Installation}
In order to install this project all that is needed are the supplementary programs described below and cloning from our repo.
\subsection{Running Project}
After everything is downloaded and code has been implemented it is required to compile and import the data into the game. In order to compile the game you need to go into ExampleAIModule in the Project folder. Then you need to run ExampleAIModule.vcxproj. When the project loads up change the solution configuration to Release not Debug In the Solution Explorer window, right click ExampleAIModule and select Build and check to see it was built correctly. Go back to the Project folder and go into the Release folder. Here you will find ExampleAIModule.dll, copy it. Go to where StarCraft Brood War is installed and find or create a folder called bwapi-data and inside it find or create a folder called AI. Inside the AI folder paste the copied file ExampleAIModule.dll and start up ChaosLauncher found in BWAPI. Make sure to run ChaosLauncher as administrator. In ChaosLauncher make sure BWAPI Injector(1.16.1)RELEASE is checked and highlighted then select config. In the config menu make sure the path for ai goes to the path where you pasted the .dll file. Now go back to the ChaosLauncher and click Start then create a custom game and it should be set.
\subsection{Requirements}
In order to install the project and correctly use it it is necessary to download other supporting software as well. It is also preferable to be working on a device running at least Windows 7. Below is a list of the required programs to be installed:\\
\\
StarCraft Brood War\\
Must be updated to version 1.16.1\\
Visual Studio Community 2013\\
Can also be Visual Studio Community 2015 with 2013 compiler tools\\
BWAPI 4.1.2\\
BWTA 1.7.1\\
\\
Important! Blizzard has taken down their installer for the original game without any patches and replaced it with version 1.18 included which does NOT support this API. As of late, there is no official installer that can be used to get version 1.16.1.
\subsection{API}
\subsubsection{ResourceGathering}
Purpose:\\
The purpose of the ResourceGathering class is to start the game and have functions pertaining to the economy of the game. The class contains basic functions and variables to keep the gathering of minerals and gasses efficient.\\
\\
Functions:\\
buildWorker\\
Input\\
Command Center\\
Processes\\
If the Command Center is idle and fails to construct a worker more supply will be built. Also, if supply is within 4 of the maximum supply, the AI has enough minerals, and it is more than 3 minutes into the game, then more supply will be built.\\
Output\\
Either a worker is made, a supply depot is made, or nothing happens.\\
\\
workerGather\\
Input\\
Command Center\\
Processes\\
Uses type of Command Center to determine race. Selects a worker to perform construction of supply structure.\\
Output\\
Worker is selected to be able to perform processes.\\
\\
gatherGas\\
Input\\
Worker\\
Processes\\
Commands the worker to gather gas.\\
Output\\
Returns true if the worker is correctly directed to gather gas and false otherwise.\\
\\
gatherMinerals\\
Input\\
Worker\\
Processes\\
Commands the worker to gather minerals.\\
Output\\
Returns true if the worker is correctly directed to gather minerals and false otherwise.\\
\\
getMineralCount\\
Input\\
None\\
Processes\\
Gets in-game amount of minerals.\\
Output\\
Count of minerals.\\
\\
getGasCount\\
Input\\
None\\
Processes\\
Get in-game amount of gas.\\
Output\\
Count of gas.\\
\\
Variables:\\
The only variables for the class is optimum gatherers for both minerals and gas and the current gatherers for minerals and gas.

\subsubsection{Building Construction}
Purpose:\\
The purpose of this class is to build any buildings necessary to play StarCraft. Any other buildings can be included in this class following this same pattern.\\
\\
Functions:\\
buildCenter\\
Input\\
Command Center, Building location(optional), player flag\\
Processes\\
Uses type of Command Center to determine race. Selects a worker to perform construction of an expansion.\\
Output\\
Command Center is built.\\
\\
buildSupply\\
Input\\
Command Center, player flag\\
Processes\\
Uses type of Command Center to determine race. Selects a worker to perform construction of supply structure.\\
Output\\
Supply depot is built\\
\\
buildGas\\
Input\\
Command Center, player flag\\
Processes\\
Uses type of Command Center to determine race. Selects a worker to perform construction of gas structure.\\
Output\\
Gas structure is built.\\
\\
buildBarracks\\
Input\\
Command Center, player flag\\
Processes\\
Build Terran Barracks if the input is of Terran type.\\
Output\\
Barracks is built.\\
\\
buildGateway\\
Input\\
Command Center, player flag\\
Processes\\
Build Protoss Gateway if the input is of Protoss type.\\
Output\\
Gateway is built.\\
\\
buildSpawningPool\\
Input\\
Command Center, player flag\\
Processes\\
Build Zerg Spawning Pool if the input is of Zerg type.\\
Output\\
Spawning Pool is built.\\
\\
checkConstructionStarted\\
Input\\
Player flag\\
Processes\\
Check construction flags and switch them off accordingly.\\
Output\\
Construction status is checked.\\
\\
Variables:\\
None\\
\\
\subsubsection{UnitAction}
Purpose:\\
The purpose of this class is to be able to control units individually to perform actions. This includes buildings as well as other movable units.\\
\\
Functions:\\
checkUnitState\\
Input\\
Unit whose state is being validated.\\
Processes\\
Checks the state of the input unit.\\
Output\\
True if unit is not dead, being constructed, disabled\\
\\
trainMarines\\
Input\\
Barracks to train marines, player flag\\
Processes\\
Starts process to train marines.\\
Output\\
Marines are trained.\\
\\
selectArmy\\
Input\\
None\\
Processes\\
Selects all marines.\\
Output\\
Marines are store in a queue.\\
\\
Variables:\\
None\\

\subsubsection{PlayerInfo}
Purpose:\\
The purpose of this class is to store variables to keep track of the game state that would otherwise be a global variable.\\
\\
Functions:\\
adjustMineralOffset\\
Input\\
player flag\\
Processes\\
Adjusts the mineral offset to store the process of building until it is actually built.\\
Output\\
Minerals are adjusted.\\
\\
adjustGasOffset\\
Input\\
player flag\\
Processes\\
Adjusts the gas offset to store the process of building until it is actually built.\\
Output\\
Gas is adjusted.\\
\\
resetMinerals\\
Input\\
player flag\\
Processes\\
Reset the mineral offset to zero.\\
Output\\
Offset is reset.\\
\\
resetGas\\
Input\\
player flag\\
Processes\\
Reset the mineral offset to zero.\\
Output\\
Offset is reset.\\
\\
Variables:\\
Flags to store what buildings are being built, counts of our current barracks and expansions, and offsets in the resource counts for planned buildings.\\

\subsubsection{MapTools}
Purpose:\\
The purpose of this class is to convert the map being played into data for the AI to interpret and use.\\
\\
Functions:\\
getNextExpansion\\
Input\\
None\\
Processes\\
Finds next resource location to build an expansion.\\
Output\\
The tile position to build the base.\\
\\
getAbsoluteTileDistance\\
Input\\
Tiles at starting and ending locations.\\
Processes\\
Calculates absolute distance not considering terrain.\\
Output\\
Distance\\
\\
Variables:\\
None\\

\subsubsection{ScoutManager}
Purpose:\\
The purpose of this class is to send out units to scout and be have units attack enemies that are seen.\\
\\
Functions:\\
scoutStartLocations\\
Input\\
Unit that will be sent to scout.\\
Processes\\
Checks all possible base starting locations starting with the furthest away from your starting base.\\
Output\\
Reference to the scout.\\
\\
attackTarget\\
Input\\
Unit that was sent to scout and the target to attack\\
Processes\\
Attacks the target with the scouting unit.\\
Output\\
Unit attacks target.\\
\\
Variables:\\
Flag to see if the unit is scouting and the unit that is scouting.\\
\\
Below is a website providing a tutorial and documentation for BWAPI:\\
http://www.teamliquid.net/blogs/485544-intro-to-scbw-ai-development\\
https://bwapi.github.io/annotated.html
\section{New Technology}
	This project helped us familiarize ourselves with new API's and the process of developing a library for other developers. Usually in an academic setting, no one else is using your code. This gave us a new perspective. We were not writing code to just solve a problem, but instead helping solve a problem in an easy to understand way as well as design it such that it may be useful to solve even greater problems. We had to think of what a developer might want to do with our API before we even considered how we would write wrappers for the other API's. 
	The primary resource used was: http://www.starcraftai.com/wiki/Main\_Page. This web page encapsulated nearly everything the team needed with links to all of the tools we used. Additionally, when choosing our development tools and language, we consulted this page because it provides information on different technologies to use when interacting with Starcraft AI. From here, the team was able to determine what best fit the needs of the project and made it easy to know what tools were available that should be considered as the project progressed.
	The team did not consult many books on this topic. For the most part, the team used a small amount of previous knowledge from AI coursework and leveraged online sources. The same goes for on-campus resources. Only one or two on-campus resources were used sparingly if at all. 

\section{What was learned}
\subsection{Brandon Chatham}
I mostly familiarized myself with machine learning, data mining, and the process of developing a library. There was a great deal of trial and error getting the tools configured and ready to use. This caused me to reach out to developers of both API's we used more than once. I gained a better idea of what it actually means to create industry level code as I was comparing my implementations and ideas to those of sophisticated AI modules that were developed by professional software developers for a competitive setting. 
As for non-technical experience gleaned from this project, it helped me learn how to effectively talk about my skills as a developer and best explain my previous work. I have found that I do a much better job of conveying my technical ideas in interviews and I think that is largely due to the confidence this project has given me. While it was not a huge up-taking, I was constantly trying to fix something or familiarize myself with something new and this now gives me the confidence that if I need to pick something up for a job, it will be a piece of cake. 
Project work is difficult. I found myself going through periods of not being able to make much progress and other times making leaps in implementations. It was extremely tedious trying to configure the API tools at times but it was very important that I figured those issues out in a timely fashion considering the team needs the code to work properly if progress is going to be made. It did not teach me much about project work that I hadn't learned from playing soccer (your team relies on you, encourage them, etc.) but it was more practice to prepare me for working in the real world comparable to what my internships had given me. 
Project management requires you to wear many hats. You need to plan for the future while ensuring the current tasks are being taken care of. Furthermore, you need to understand the situations of your teammates who are working on the project and evaluate what is actually realistic considering their workload, personal life, etc.. 
Working in teams is nothing new for me. Playing 20 years of organized sports has given me one of my best skills: seamlessly coming into a team and doing my best to find my fit while applying my skill set as effectively as I can without intruding on others. My first intention is to work hard and communicate effectively. If a leadership role arises and it fits me like it did on this project, I'm happy to take it. This was another opportunity to develop my skills in this area.
If I could do it all over, I would spend more time trying to figure out how to create a basic neural network for our library. That would be have been an outstanding tool for new developers to familiarize themselves with and would have been awesome to explain to employers how we did it.
\subsection{Kristen Patterson}
The technical information I learned is developing a library for a large project, how to properly organize and document a large project, and the intricacies of machine learning. It also put a great importance on time management as even setting up the environment to program was such a lengthy task it could much more time than I originally thought. Since we were building a library, I also had to frequently search through documentation and try and decipher what each function use was and check to see if there was a more efficient alternative. The practice in machine learning also helped me to realize the numerous amount of conditionals that could be found when trying to replace a human with an agent.
The non-technical information I learned was the experience i have taken away for a project as big as this. I have not really worked on projects as large as this and a lot of it was easily translated into other fields. Since I had worked on this project for so long and talked about it continually for months on end I was able to describe it in any level of complexity. This translates well in being able to talk about any project or have confidence in any of my methods. In order to be able to answer quickly and show that I understand my field clearly I need to just spend time to voice my projects and actions.
Group work and project work require a large amount of time and dedication. When working on projects and in a group it is important to spend time organizing and try to make a schedule and keep it. It is also important to be able to designate time and resources appropriately before any work even begins. I have also realized that projects require a level of compromise and being able to give features and functions a priority based on how necessary they are to a projects success.
Managing a team and project can be fairly difficult. Not only do problems show up in the form of bugs or realizing functions will not be able to be delivered, but they can also be in the form of lack of time to complete certain functions and not being able to accurately distribute resources to parts of a project. It also requires being able to understand the group's strengths and where they are best utilized.
Interaction with teammates can be slightly hectic. As an individual apart of a team it is important to be able to show empathy and understanding for any misgivings that may occur in a teammates life. It is also important to not try to burden teammates with large workloads due to issues on your end. While rough, it is also important to try and keep the group on task and motivate them or provide help with their work.
If I could redo this whole experience I would have mainly tried to focus more on the core requirements of the project before focusing on the stretch goals. I would have also liked to give myself more free time to work on the project in the winter by taking a lighter class load. If I had more time I would have been able to start on the actual artificial intelligence portion of the code as that would have made our project more attractive.
\end{document}
