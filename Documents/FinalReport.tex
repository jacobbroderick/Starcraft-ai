\documentclass[10pt,letterpaper,onecolumn,draftclsnofoot]{IEEEtran}
\usepackage[margin=0.75in]{geometry}
\usepackage{listings}
\usepackage{color}
\usepackage{longtable}
\usepackage{tabu}
\definecolor{dkgreen}{rgb}{0,0.6,0}
\definecolor{gray}{rgb}{0.5,0.5,0.5}
\definecolor{mauve}{rgb}{0.58,0,0.82}
\newcolumntype{P}{>{\centering\arraybackslash}m{6cm}}
\newcolumntype{L}{>{\centering\arraybackslash}m{3cm}}
\newcolumntype{C}{>{\centering\arraybackslash}m{3cm}}
\newcolumntype{B}{>{\centering\arraybackslash}m{1.5cm}}
\newcolumntype{M}{>{\centering\arraybackslash}m{2.5cm}}

\lstset{frame=tb,
	language=C,
	columns=flexible,
	numberstyle=\tiny\color{gray},
	keywordstyle=\color{blue},
	commentstyle=\color{dkgreen},
	stringstyle=\color{mauve},
	breaklines=true,
	breakatwhitespace=true,
	tabsize=4
}

\begin{document}
	
	\begin{titlepage}
		
		\title{CS 463 - Spring 2017 - Final Report \\ Project DevAI}
		\author{Jacob Broderick, Kristen Patterson, Brandon Chatham}
		\date{June 2, 2017}
		\maketitle
		\vspace{4cm}
		\begin{abstract}
			This document details the progress made by the Starcraft AI Project. It includes a brief recap of the project, where the project currently is, and what stretch goals remain unfinished. It also talks about any problems that have occurred. This document was made with a video that details similar information, but demonstrates what the working project looks like.
		\end{abstract}
	\end{titlepage}

\tableofcontents
\newpage
\section{Introduction}
	This project was requested by our client Dr. Fern. Dr. Fern's primary intent for this project was for us to reduce the learning curve of artificial intelligence and machine learning implementations for an AI club that will soon be created at OSU. Learning AI can be difficult but the basics can be picked up quickly. Some of the trouble spots for new developers who are not well-versed in the process of starteing projects might get hung up on configuring API's and setting up their development environment. Our project handles the configuration step for them. The environment is ready to use as soon as they open the Visual Studios project. Furthermore, if anything gets changed in a developer's settings, we have thorough documentation explaining how to fix it. Beyond this, we have included tools to help more skilled developers start data mining for machine learning algorithms with a python script we wrote and a replay analyzer we researched and chose from an open source developer. 
	There are three team members: Brandon Chatham, Kristen Patterson, and Jacob Broderick. Brandon primarily focused on code design, implementations, and overall direction of the project to best fit the desires of the client. Kristen focused on documentation. Nearly all of the documentation, even a small bit of the comments you will find in the code, are written by Kristen. Kristen worked closely with Brandon to keep up-to-date on what the newest implementations were in order to keep documentation relevant to the current state of the project. Jacob focused mostly on choosing our development tools in Visual Studios and BWAPI as well as researched other tools that may be valuable to us. BWAPI is the standard API used for Starcraft AI and was an easy choice. Determining what other tools were well-suited to the specifications of the project were slightly more difficult but fairly simple decisions. 
	Our client for the most part gave us autonomy on the project. While we did meet with him for required check-in's for the project, he allowed us to make our own decisions whether this was for better or for worse. We did consult him on a few occasions but usually made decisions primarily based on what the team felt was best based on what we knew Dr. Fern was most interested in.
\section{Original Requirements Document}

\section{Changes from Original Requirements Document}

\section{Design Document}

\section{Tech Review}

\section{Weekly Blog Posts}

\section{Final Poster}

\newpage

\section{Project Documentation}

\section{New Technology}
	This project helped us familiarize ourselves with new API's and the process of developing a library for other developers. Usually in an academic setting, no one else is using your code. This gave us a new perspective. We were not writing code to just solve a problem, but instead helping solve a problem in an easy to understand way as well as design it such that it may be useful to solve even greater problems. We had to think of what a developer might want to do with our API before we even considered how we would write wrappers for the other API's. 
	The primary resource used was: http://www.starcraftai.com/wiki/Main\_Page. This web page encapsulated nearly everything the team needed with links to all of the tools we used. Additionally, when choosing our development tools and language, we consulted this page because it provides information on different technologies to use when interacting with Starcraft AI. From here, the team was able to determine what best fit the needs of the project and made it easy to know what tools were available that should be considered as the project progressed.
	The team did not consult many books on this topic. For the most part, the team used a small amount of previous knowledge from AI coursework and leveraged online sources. The same goes for on-campus resources. Only one or two on-campus resources were used sparingly if at all. 

\section{What was learned}
\subsection{Brandon Chatham}
I mostly familiarized myself with machine learning, data mining, and the process of developing a library. There was a great deal of trial and error getting the tools configured and ready to use. This caused me to reach out to developers of both API's we used more than once. I gained a better idea of what it actually means to create industry level code as I was comparing my implementations and ideas to those of sophisticated AI modules that were developed by professional software developers for a competitive setting. 
As for non-technical experience gleaned from this project, it helped me learn how to effectively talk about my skills as a developer and best explain my previous work. I have found that I do a much better job of conveying my technical ideas in interviews and I think that is largely due to the confidence this project has given me. While it was not a huge up-taking, I was constantly trying to fix something or familiarize myself with something new and this now gives me the confidence that if I need to pick something up for a job, it will be a piece of cake. 
Project work is difficult. I found myself going through periods of not being able to make much progress and other times making leaps in implementations. It was extremely tedious trying to configure the API tools at times but it was very important that I figured those issues out in a timely fashion considering the team needs the code to work properly if progress is going to be made. It did not teach me much about project work that I hadn't learned from playing soccer (your team relies on you, encourage them, etc.) but it was more practice to prepare me for working in the real world comparable to what my internships had given me. 
Project management requires you to wear many hats. You need to plan for the future while ensuring the current tasks are being taken care of. Furthermore, you need to understand the situations of your teammates who are working on the project and evaluate what is actually realistic considering their workload, personal life, etc.. 
Working in teams is nothing new for me. Playing 20 years of organized sports has given me one of my best skills: seamlessly coming into a team and doing my best to find my fit while applying my skill set as effectively as I can without intruding on others. My first intention is to work hard and communicate effectively. If a leadership role arises and it fits me like it did on this project, I'm happy to take it. This was another opportunity to develop my skills in this area.
If I could do it all over, I would spend more time trying to figure out how to create a basic neural network for our library. That would be have been an outstanding tool for new developers to familiarize themselves with and would have been awesome to explain to employers how we did it.
\end{document}
