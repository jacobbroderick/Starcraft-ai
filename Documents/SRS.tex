\documentclass[10pt,letterpaper,onecolumn,draftclsnofoot]{IEEEtran}
\usepackage[margin=0.75in]{geometry}
\usepackage{listings}
\usepackage{color}
\usepackage{longtable}
\usepackage{tabu}
\usepackage{pgfgantt}
\definecolor{dkgreen}{rgb}{0,0.6,0}
\definecolor{gray}{rgb}{0.5,0.5,0.5}
\definecolor{mauve}{rgb}{0.58,0,0.82}

\lstset{frame=tb,
  language=C,
  columns=flexible,
  numberstyle=\tiny\color{gray},
  keywordstyle=\color{blue},
  commentstyle=\color{dkgreen},
  stringstyle=\color{mauve},
  breaklines=true,
  breakatwhitespace=true,
  tabsize=4
}

\begin{document}

\begin{titlepage}

  \title{CS 461 - Fall 2016 - Requirements Document \\ Project DevAI}
  \author{Jacob Broderick, Kristen Patterson, Brandon Chatham}
  \date{October 28, 2016}
  \maketitle
  \vspace{4cm}

\end{titlepage}

\tableofcontents
\newpage

\section{Introduction}
This section covers the purpose and the overview of this document. It also includes a list of abbreviations and definitions as well as describes the scope of the project. Any sources referenced throughout the document are also included.

\subsection{Purpose}
The purpose of this document is to describe the requirements of "Project DevAI". This document will also describe the purpose of the project as well as name some of the constraints that will be involved in the process of development. The intended audience for this document is the client and the development team. The client can make sure that their specific concerns and constraints have been met, while the development team can use this document as a reference during the development process.

\subsection{Scope}
"Project DevAI" is an AI that plays Starcraft Brood War and provides a user with a well-documented AI that can be used for research purposes. The AI should implement at least one strategy to play Starcraft Brood War. A user can interact with the AI by opening the source code and viewing it. The source code must be highly documented and modular that way users can pull any section of code from the source code and use it for their own program.

\subsection{Definitions, Acronyms, and Abbreviations}
AI - artificial intelligence
Agent - the program that will play Starcraft Brood War by itself
Client - person or organization that has requested the project to be developed
User - any future Oregon State University club member that wishes to interact with the artificial intelligence
Starcraft Brood War - a real time strategy game created by Blizzard Entertainment

\subsection{References}
[1] IEEE Software Engineering Standards Committee, “IEEE Std 830-1998, IEEE Recommended Practice for Software Requirements Specifications”, October 20, 1998.

\subsection{Overview}
This document contains two more sections. The second sections contains information about the product perspective and functions. It also details the user characteristics and discusses the various constraints and dependencies the program has. The third section provides the specific and functional requirements of the system.

\newpage

\section{Overall Description}

\subsection{Product Perspective}

\subsection{Product Functions}

\subsection{User Characteristics}

\subsection{Constraints}

\subsection{Assumptions and Dependencies}

\newpage

\section{Specific Requirements}
The agent should be comprised of modular components that are well-documented. Functionality should be broken into stand-alone pieces that could nearly directly translate into another developer's agent. Also, these pieces should be documented well enough such that a developer can understand and make adjustments to them per their needs. Our stretch goal is to give the agent the ability to switch between scripts based on game circumstances. This would mean adjusting to a new strategy based on the agent's resources and potentially the opponents estimated resources.

\subsection{Functional Requirements}
Our agent will use the BWAPI to interact with the Starcraft Brood War game. Our agent will follow a scripted strategy. Since the intent is for the agent to play the game on its own, there is no user interface beyond running the program. People can potentially play against the agent, however this requires no additional interaction with our agent. The scripted strategies will be based on how we can engineer the agent to gather resources and expend them while also eliminating the opponent's ability to do these things.

\begin{figure}[h]
\begin{center}

\begin{ganttchart}[y unit title=0.4cm,
y unit chart=0.5cm,
vgrid,hgrid, 
title label anchor/.style={below=-1.6ex},
title left shift=.05,
title right shift=-.05,
title height=1,
bar/.style={fill=gray!50},
incomplete/.style={fill=white},
progress label text={},
bar height=0.7,
group right shift=0,
group top shift=.9,
group height=.3,
group peaks height=.2]
{1}{30}
%labels
\gantttitle{Terms}{30} \\
\gantttitle{Fall 2016}{10} 
\gantttitle{Winter 2017}{10} 
\gantttitle{Spring 2017}{10} 
%tasks
\ganttbar{Problem Statement}{3}{5} \\
\ganttbar[progress=50]{Requirements Document}{4}{6} \\
\ganttbar[progress=0]{Technical Document}{7}{9} \\
\ganttbar[progress=0]{Design Document}{7}{9}

\end{ganttchart}
\end{center}
\caption{Gantt Chart}
\end{figure}

\end{document}
