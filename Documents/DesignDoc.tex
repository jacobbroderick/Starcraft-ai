\documentclass[10pt,letterpaper,onecolumn,draftclsnofoot]{IEEEtran}
\usepackage[margin=0.75in]{geometry}
\usepackage{listings}
\usepackage{color}
\usepackage{longtable}
\usepackage{tabu}
\definecolor{dkgreen}{rgb}{0,0.6,0}
\definecolor{gray}{rgb}{0.5,0.5,0.5}
\definecolor{mauve}{rgb}{0.58,0,0.82}
\newcolumntype{L}{>{\centering\arraybackslash}m{6cm}}
\newcolumntype{C}{>{\centering\arraybackslash}m{3cm}}
\newcolumntype{B}{>{\centering\arraybackslash}m{1.5cm}}
\newcolumntype{M}{>{\centering\arraybackslash}m{2.5cm}}

\lstset{frame=tb,
  language=C,
  columns=flexible,
  numberstyle=\tiny\color{gray},
  keywordstyle=\color{blue},
  commentstyle=\color{dkgreen},
  stringstyle=\color{mauve},
  breaklines=true,
  breakatwhitespace=true,
  tabsize=4
}

\begin{document}

\begin{titlepage}

  \title{CS 461 - Fall 2016 - Design Document \\ Project DevAI}
  \author{Jacob Broderick, Kristen Patterson, Brandon Chatham}
  \date{December 2, 2016}
  \maketitle
  \vspace{4cm}
  \begin{abstract}
  	\noindent 
  	  	\noindent
	The goal of this project is to create an agent to play the game Starcraft Brood War, a real time strategy game created by Blizzard Entertainment. The solution to the project will be a template that future students can use to develop better solutions than those provided in this project. In this document, we will discuss how we will design and implement our project. The design choices we make will help as a road map and checklist to follow.
   \end{abstract}
\end{titlepage}

\tableofcontents

\newpage

\section{Introduction}
\subsection{Date of issue and status}

\subsection{Scope}

\subsection{Issuing organization}

\subsection{Authorship}

\subsection{References}

\subsection{Context}

\subsection{Design languages}

\subsection{Body}

\subsection{Summary}

\subsection{Glossary}

\subsection{Change history}


\section{Design Stakeholders}
	A major stakeholder in the design process is future students involved in an Oregon State University club. These students will be reading and using the code included in the project.
	
\section{Design concerns}
\subsection{Documentation}
	The design concerns of the future students are that the code is well documented and organized. 
\subsection{Modularity}
	In order for the code to be transferable, it should be broken into modular pieces focused on specific functionality accessible through the API.
\section{Design Views}
\subsection{Documentation View}
	The code is well documented if it has concise comments and includes functional header comments. In order for the code to be organized, it must be modularized into simple functions and further separated into header files containing functions with similar purposes. The functions also need to be listed in order of complexity with the most complex functions requiring the highest amount of dependencies is located at the bottom.
\subsection{Modularity}
	The code will be modular with respect to the functionality of the BWAPI. Creating stand-alone modules that perform necessary Starcraft game actions will improve the learning process for students who use the code, and will enable them to produce AI agents more quickly even with limited or no experience with AI software design. Furthermore, the modules should have limited or no dependency on other modules. This allows for simple transferability into a developer's code. 
\section{Design Viewpoints}
\subsection{Documentation}
\subsubsection{Pattern}
	The pattern viewpoint covers the design for the commenting of the code. Commenting in the program will require reuse of function header comments to describe the function's purpose. Functional header comments will follow a specified template of describing the name, process, inputs, outputs, and requirements of a function. In-line comments must be included for any variable declarations, loops, and function calls as well as any ambiguous declarations. The parts of the functional header comment must directly associate with code in the function. For example, the inputs are the parameters of a function and the outputs are the returned values. This viewpoint will be supported using the UML composite structure diagram.
	
\subsubsection{Composition}
	The composition viewpoint is used for the design of modular functions and files. Composition will help to localize and simplify blocks of code into functions. Functions will then be localized into separate files based on their purpose. Most functions will be decomposed into modules with a singular purpose while their will be a few functions that control the system and call upon the simpler modules. The HIPO diagram will be used to supplement this viewpoint.

\subsubsection{Dependency}
	The dependency viewpoint designs the order of the functions in each file. The design of each file will follow the order of least complex function with least dependencies on top while more complex functions that use other functions are on the bottom of the file. A UML component diagram will be used to show the dependencies.
\subsection{Modularity}
\subsubsection{Structure}
	The design concerns of modularity are code reusability, transferability, and understandability. Modularity of the library will be categorized with respect to individual classes or small groups of classes relevant to the desired functionality. The API classes are already broken into categories for interacting with the AI agent and game. This library will access functions relevant to specific actions based on functionality of the module. This design convention makes learning easier by providing concise examples for developers to use and study.
	An example would be pulling game state information with regards to resources. This will be important for many of the logic modules created. These modules will organize the functions that pull the appropriate data and can send it to other modules for further algorithm-based actions. In this example of resource collection, modularity makes it easier for students to separate the logic of data gathering from the other categories of functionality. Additionally, it makes modules more reusable and transferable without disrupting other systems within the overall body of code.

\section{Design Overlays}
\subsection{Modularity}
	As mentioned in the Requirements Document, we will be breaking categories into micro and macro game actions. Micro actions are tactical and more unit-specific while macro actions are resource gathering and base construction focused. Also, these categories are broken into decision-making modules that pull game data and algorithmically decide what should be done and communicate with action modules that act on the decision making. This is applicable to the modularity design concerns as it provides developers easier code sharing and transferability when developing competitive AI's with other developers. 
\section{Design Rationale}
\subsection{Documentation}
\subsubsection{Commenting}
	Adding function header comments and using a template will help future students to quickly and easily find the information they need about the function. In-line comments will also help future students follow the process in the function.

\subsubsection{Modular functions and files}
	Keeping functions modular and organized into separate files will help future students to easily navigate the code. It will also help them use the code as the can take separate modules or files easily.

\subsubsection{Ordered functions}
	Ordering functions by complexity will help the future students navigate as they can quickly find simple functions to use. They will also be able to physically see the use of simple functions to support larger more complex functions.
\subsection{Modularity}
\subsubsection{Categorization of Modules}
	Categorizing the library modules first based on the functionality they serve with respect to the game, then micro and macro game actions, and finally decision-making or action will make library components concise, easier to understand, and more transferable across AI agents.
\end{document}