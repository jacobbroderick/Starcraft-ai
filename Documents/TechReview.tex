\documentclass[10pt,letterpaper,onecolumn,draftclsnofoot]{IEEEtran}
\usepackage[margin=0.75in]{geometry}
\usepackage{listings}
\usepackage{color}
\usepackage{longtable}
\usepackage{tabu}
\definecolor{dkgreen}{rgb}{0,0.6,0}
\definecolor{gray}{rgb}{0.5,0.5,0.5}
\definecolor{mauve}{rgb}{0.58,0,0.82}
\newcolumntype{L}{>{\centering\arraybackslash}m{6cm}}
\newcolumntype{C}{>{\centering\arraybackslash}m{3cm}}

\lstset{frame=tb,
  language=C,
  columns=flexible,
  numberstyle=\tiny\color{gray},
  keywordstyle=\color{blue},
  commentstyle=\color{dkgreen},
  stringstyle=\color{mauve},
  breaklines=true,
  breakatwhitespace=true,
  tabsize=4
}

\begin{document}

\begin{titlepage}

  \title{CS 461 - Fall 2016 - Requirements Document \\ Project DevAI}
  \author{Jacob Broderick, Kristen Patterson, Brandon Chatham}
  \date{October 28, 2016}
  \maketitle
  \vspace{4cm}
  \begin{abstract}
  	\noindent 
  	  	\noindent
	This project is to create an agent to play the game Starcraft Brood War, a real time strategy game created by Blizzard Entertainment. After this project is complete, it could be expanded upon by future students in a club setting. This project will be a template so that future students can strive to develop better solutions than those developed within this project. This document will provide the details of the project as well as provide a list of tasks the development team must complete. The goal of this document is to specify the requirements for the system to be developed.
   \end{abstract}
\end{titlepage}

\tableofcontents

\newpage

\section{Introduction}

\section{Technologies}

\subsection{Programming paradigm for functions}
\subsubsection{Options}
\begin{center}
\begin{tabular}{ ||C|L|L|| } 
\hline
Paradigm & Description & Reason for Selection \\
 \hline
 Modular & Paradigm focused on separating functions into independent modules. & Modular has a high focus on independence of functions. \\ 
 \hline
 Object Oriented & Paradigm focused on objects such as data and attributes rather than actions. & Objects can help to section off parts of the program. \\ 
 \hline
 Recursion & Paradigm focused on solving a larger problem using solution from multiple smaller problems. & The program requires the solution of many smaller sections. \\ 
 \hline
\end{tabular}
\end{center}

\subsubsection{Goals}
The goal of comparing the chosen programming paradigms for functionalization is to clarify which method of coding is best for our project. It will also help to remind us of how functions need to be set up and whether a section of code needs to be broken down into separate blocks.

\subsubsection{Criteria}

\subsubsection{Table}

\subsubsection{Discusion}

\subsubsection{Selection}
The client specifically requested that we have our code be modular enough that a coder can grab a function and use it in their program with little to no problem. Keeping the system modular will also help with documenting the code and keeping it well-commented.

\subsection{Programming language}
\subsubsection{Options}
\begin{center}
\begin{tabular}{ ||C|L|L|| } 
\hline
Language & Description & Reason for Selection \\
 \hline
 C++ & Commonly used programming language with many functionalities derived from C. & It is commonly taught and used and is the recommended language from BWAPI. \\ 
 \hline
 Java & Class-based programming language that promotes few implementation dependencies. & It can be run on many different systems.\\ 
 \hline
 C Sharp & Programming language derived from C mainly used for object oriented code. & It has a high focus on objects while keeping a lot of the functions of C. \\ 
 \hline
\end{tabular}
\end{center}

\subsubsection{Goals}
Comparing languages will help to settle the environment as well as the core of our assignment. Choosing a language will also help decide the programming practices and format we will decide on.

\subsubsection{Criteria}

\subsubsection{Table}

\subsubsection{Discusion}

\subsubsection{Selection}
Due to the client?s request, we will be choosing to work in C++. This language will also help in modularizing our system as well as staying within the tournament?s guidelines for performance.

\subsection{Environment and IDEs}
\subsubsection{Options}
\begin{center}
\begin{tabular}{ ||C|L|L|| } 
\hline
IDE & Description & Reason for Selection \\
 \hline
 Visual Studio[1] & IDE created by Microsoft that can create Windows programs.  & Highly popular IDE for C++ and C Sharp and recommended by BWAPI. \\ 
 \hline
 Eclipse[2] & Cross-platform IDE mainly used for Java. Eclipse requires plugins in order to customize the environment. & Popular IDE for both Java and C++ projects. \\ 
 \hline
 IntelliJ IDEA[3] & Java integrated IDE that supports cross-platform and was created by JetBrains.
 & Thorough tools and development centered around Java. \\ 
 \hline
\end{tabular}
\end{center}

\subsubsection{Goals}
The goal for the comparison of different IDEs is to distinguish a sleek efficient environment for the developers. The IDE must also provide tools to increase the speed of debugging and must work correctly with the language we select to work with.

\subsubsection{Criteria}

\subsubsection{Table}

\subsubsection{Discusion}

\subsubsection{Selection}
Since we are choosing to program in C++, and it is the recommended environment for the BWAPI, we will be working within Visual Studio.

\subsection{APIs}
\subsubsection{Options}
\begin{center}
	\begin{tabular}{ ||C|L|L|| } 
		\hline
		API & Description & Reason for Selection \\
		\hline
		BWAPI & BWAPI designed to build AI agents. & Includes functionality to automate necessary game mechanics and use information on the current state of the game. \\ 
		\hline
		Brood Data API & API focused on data mining for future use in machine-learning algorithms for AI or competitive player training. & Useful for gathering data for design algorithms.\\ 
		\hline
		 BW Spectator Interface API & API designed for spectator-mode interface customization with real-time game stats and interaction with units. & Useful for observing performance of AI. \\ 
		\hline
	\end{tabular}
\end{center}

\subsection{Documentation for API}
\subsubsection{Options}
\begin{center}
	\begin{tabular}{ ||C|L|L|| } 
		\hline
		Source & Description & Reason for Selection \\
		\hline
		BWAPI  & Commonly used programming language with many functionalities derived from C. & It is commonly taught and used and is the recommended language from BWAPI. \\ 
		\hline
		BWAPI Wiki & Web page with links to BWAPI source code documentations as well as other links to other BWAPI related information such as extensions, tutorials and game fundamentals. & Useful for identifying and explaining all of the tools available to the project. \\ 
		\hline
		Source Code Comments & Explanations of code functionality directly within the code. & Concise explanation of code located inside of the code for easy access. \\
		\hline
	\end{tabular}
\end{center}

\subsection{Choice algorithm}
\subsubsection{Options}
\begin{center}
	\begin{tabular}{ ||C|L|L|| } 
		\hline
		Type & Description & Reason for Selection \\
		\hline
	Scripted Tree Selection & Creating a tree with scripted strategies to alternate between in response to game data. & Significantly easier implementation that still provides flexibility between strategies. \\ 
		\hline
		Machine Learning - Supervised & AI would base decisions on outcomes of previous instances of decisions and those outcomes. & Identifies patterns in outcomes and can eventually identify the appropriate response based on previous successes or mistakes. \\ 
		\hline
		Machine Learning - Reinforcement Learning & Creates a reward system for which the program will attempt to maximize reward by responding to each point of data given. & Powerful when making decisions and evaluating the effectiveness of that decision given the current state of the game. \\ 
		\hline
	\end{tabular}
\end{center}

\subsection{Extensions to the API}

\subsection{Compiling environment}

\subsection{Compiling type}

\section{Conclusion}

\newpage

\section{Bibliography}

\end{document}
